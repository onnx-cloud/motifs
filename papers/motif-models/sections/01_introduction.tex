% Section 1: Introduction

\section{Introduction}
\label{sec:intro}

Computation graphs form the backbone of modern machine learning systems.
From TensorFlow and PyTorch to ONNX and specialized accelerators, these
intermediate representations (IRs) must be analyzed, optimized, and verified
for correctness.
Yet existing tools treat graphs as monolithic structures, lacking a principled vocabulary
for discussing recurring patterns.

We propose a systematic \emph{taxonomy of computation-only graph motifs}---composable,
reusable patterns that capture the essential structure of graph computation.
Motifs enable:

\begin{itemize}
  \item \textbf{Equivalence reasoning}: Prove that different graph rewrites preserve semantics.
  \item \textbf{Transformation legality}: Define which graph manipulations are safe.
  \item \textbf{Parallelism analysis}: Reason about data-flow dependencies and bounds.
  \item \textbf{IR semantics}: Specify computation without imperative control flow.
  \item \textbf{Resource analysis}: Predict memory, latency, and determinism properties.
\end{itemize}

\subsection{Contributions}

\begin{enumerate}
  \item A \textbf{representative set} of core computation motifs, organized by category.
  \item \textbf{Canonical ONNX reference implementations} and Python examples for each motif.
  \item An \textbf{RDF/TTL ontology} defining motif terms and relationships.
  \item \textbf{Case studies} showing how real workloads decompose into motif compositions.
  \item A \textbf{verification framework} using motif structure to speed up correctness proofs.
\end{enumerate}

\subsection{Roadmap}

The remainder: \S\ref{sec:background} reviews prior work.
\S\ref{sec:methods} introduces our taxonomy and formalism.
\S\ref{sec:results} presents examples and experiments.
\S\ref{sec:discussion} discusses limitations.
\S\ref{sec:conclusion} summarizes contributions.
