% Section 2: Background and Related Work

\section{Background and Related Work}
\label{sec:background}

\subsection{Graph Motifs in Biology}

Graph motifs originate in computational biology and network science.
A motif is a recurring subgraph pattern appearing significantly more often than expected by chance.
Classical motif discovery algorithms (ESU, FANMOD) identify patterns and infer functional roles.

In computation graphs, we adopt a related but distinct perspective: rather than detecting
statistically significant patterns, we \emph{design} a catalog of semantically meaningful motifs
that capture common computation structures.

\subsection{Intermediate Representations}

Modern ML frameworks use computation graphs as IRs:
\begin{itemize}
  \item \textbf{TensorFlow}: Static and eager graphs with fused ops.
  \item \textbf{PyTorch}: Dynamic control flow; TorchScript converts to static graphs.
  \item \textbf{ONNX}: Standard IR supporting fusion, quantization, shape inference.
  \item \textbf{XLA}: Compiles graphs to optimized accelerator code via pattern matching.
\end{itemize}

Existing optimizers rely on hand-crafted rewrite rules. A motif taxonomy provides a principled,
reusable foundation for specifying these rules formally.

\subsection{Dataflow and Control Flow}

Classical dataflow analysis models computation as DAGs with fixed control flow.
Many modern workloads introduce stateful loops, conditionals, and memory patterns beyond pure dataflow.
Our taxonomy includes control flow primitives and memory/state motifs, bridging dataflow and imperative programming.

\subsection{Gaps in Current Tools}

Current tools lack:
\begin{itemize}
  \item A \emph{principled vocabulary} for discussing graph transformations across frameworks.
  \item \emph{Formal semantics} linking motif structure to rewrite legality and resource bounds.
  \item \emph{Machine-readable} definitions (ontology) of motif classes and properties.
  \item \emph{Canonical reference implementations}.
\end{itemize}
