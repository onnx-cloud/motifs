\documentclass[11pt,oneside]{article}

% Essential packages
\usepackage[utf8]{inputenc}
\usepackage[T1]{fontenc}
\DeclareUnicodeCharacter{22A5}{\ensuremath{\bot}} % map Unicode '⊥' to LaTeX \bot
\usepackage{amssymb}
\usepackage{amsmath}
\usepackage{geometry}
\usepackage{fancyhdr}
\usepackage{graphicx}
\usepackage{hyperref}
\usepackage{listings}
\usepackage{xcolor}
\usepackage{booktabs}
\usepackage{array}

% Geometry
\geometry{margin=1in, headheight=14pt}

% Bibliography (plain natbib - no biber required)
\usepackage{natbib}
\bibliographystyle{plainnat}
% Fallbacks for uncommon packages
% If stmaryrd isn't installed, define bracket macros to avoid build failure
\providecommand{\llbracket}{[\![}
\providecommand{\rrbracket}{]\!]}


% Metadata
\def\ReportTitle{Statically Typed Reality: A Type-Preserving ABI for Machine Intelligences}
\def\ReportAuthors{Anonymous}
\def\ReportDate{\today}

% Commands
\newcommand{\Tensor}[1]{\texttt{#1}}
\newcommand{\OnePageSectionStart}{}
\newcommand{\OnePageSectionEnd}{}

% Code listing style
\lstset{
  basicstyle=\ttfamily\small,
  breaklines=true,
  breakatwhitespace=true,
  columns=flexible,
  keywordstyle=\color{blue},
  commentstyle=\color{gray},
  stringstyle=\color{red},
  showstringspaces=false,
  tabsize=2
}

% Headers/footers
\pagestyle{fancy}
\fancyhf{}
\rhead{\thepage}
\lhead{\ReportTitle}


\title{\ReportTitle}
\author{\ReportAuthors}
\date{\ReportDate}

\begin{document}
\maketitle

\begin{abstract}
We present a statically typed ABI and abstract syntax tree (AST) that bridges observation and computation: sensors, symbolic knowledge, and probabilistic models are unified as typed tensors, fused through type-preserving primitives, and lowered to any runtime (ONNX, GPU kernels, hardware accelerators, or domain-specific interpreters). Each tensor is a triple---\emph{DType, Shape, Semantics}---that grounds computation in measurable reality and abstract knowledge. This enables deterministic interpretation of the world, where perception (vision, audio, language), reasoning (distributed, probabilistic, symbolic), and action (motor, communication, control) are formally connected. Test and train graphs are identical in structure and semantics, enabling reproducible evaluation and formal validation. Compile-time verification rejects type-unsafe programs; where checks remain dynamic, explicit assertions are generated and elided when provably safe.
\end{abstract}

\tableofcontents
\clearpage

% Sections
\OnePageSectionStart
\section{Introduction}

The chasm between perception and reasoning remains the primary challenge in artificial intelligence. Traditional "connectionist" models operate on untyped, high-dimensional manifolds where the semantic provenance of data is erased at the input layer \citep{pytorch2019, tensorflow2016}. Conversely, symbolic systems often lack the computational fluidity required to process high-bandwidth sensory observations \citep{nilsson1991}.

We propose a unification: the \emph{Fused Fabric}. This framework posits that the structure of reality (ontological grounding) and the structure of reasoning (computational fusion) are two aspects of a single topological object. To process an observation is to map it onto an ontological coordinate; to reason is to fuse the graph-based histories of these coordinates across heterogeneous domains.

This unification is operationalized through two distinct but complementary systems:
\begin{enumerate}
    \item \textbf{Typed Reality}: A system for static semantic grounding where every tensor is enriched with ontological metadata, turning raw numbers into physical measurements (Measurement), symbolic indices (Enum), or probabilistic beliefs (Distribution).
    \item \textbf{Compiled Cognition}: A system for graph-based reasoning that formalizes the fusion of specialized compute graphs into machine-native, silicon-optimized code (ONNX).
\end{enumerate}

By blending these fabrics, we move beyond "stochastic parrots" toward machine intelligences that interpret the world through a formal, verifiable, and silicon-efficient lens \citep{bender2021}.

\OnePageSectionEnd

\OnePageSectionStart
\section{Type System}

Traditional type systems (Hindley-Milner \citep{hindley1969}, System F \citep{girard1972}) encode properties of computation. Dependent types \citep{coq2023, agda2023} allow types to encode propositions about values. We extend this paradigm: types encode properties of \emph{meaning}. A Measurement type is not merely a numeric constraint but a statement about grounding in physical reality.

\subsection{Core Tensor Type}
A tensor is a triple:
\[ \text{Tensor}\langle \text{DType}, \text{Shape}, \text{Semantics} \rangle \]
which we express in the AST as \texttt{DType[Shape] Sem}. For example, \texttt{f32[B,128] ⊥} denotes a float32 tensor of shape $[B, 128]$ with Atomic semantics. DType $\in \{\texttt{f16}, \texttt{f32}, \texttt{i8}, \texttt{bool}, \ldots\}$. Shape is determined by a dimension list (constant, symbolic, or range). Semantics grounds meaning in observable reality or abstract knowledge.

\subsection{Semantic Domains}
\begin{itemize}
  \item \textbf{Atomic} (\texttt{⊥}): unstructured numeric substrate; generic latent vectors without semantic commitment
  \item \textbf{Measurement} (\texttt{M}): grounds tensors in physical reality via IRIs---radiance (\texttt{qudt:Radiance}), pressure (\texttt{qudt:Pressure}), temperature (\texttt{qudt:Temperature})---with unit, valid range, and observational frame (sensor IRI). Follows SOSA/SSN \citep{sosa2017} and QUDT \citep{qudt2023} standards.
  \item \textbf{Distribution} (\texttt{D}): encodes probabilistic knowledge: unnormalized (logits over support), normalized (probabilities), or samples from a stochastic process (IRI reference to Gaussian, Poisson, etc.)
  \item \textbf{Enum} (\texttt{E}): grounds tensors in symbolic knowledge via SKOS concept schemes \citep{skos2009}: discrete vocabulary (BPE tokens, class labels, entity IRIs) with fixed cardinality
  \item \textbf{Structured} (\texttt{\{...\}}): composite types combining Measurements, Distributions, and Enums; enables fusing heterogeneous knowledge (e.g., visual features + textual embedding + pose measurement)
\end{itemize}
\textbf{Key principle:} Semantics are \emph{not} erased. They flow through the computation graph and are enforced at type check and lowering time. A tensor's semantic grounding is invariant: a Measurement remains grounded in physical reality; a Distribution preserves probabilistic constraints; an Enum maintains symbolic vocabulary. This enables the system to detect semantic mismatches (e.g., attempting to average class indices) that would silently corrupt untyped systems like PyTorch \citep{pytorch2019} or TensorFlow \citep{tensorflow2016}.

\subsection{Subtyping and Compatibility}
Edge compatibility enforces the fusion of knowledge realms. When connecting two morphs, the system checks:
\begin{enumerate}
  \item \textbf{DType match}: no implicit numeric casts; \texttt{f32} fuses with \texttt{f32}, not \texttt{i32}
  \item \textbf{Shape unification}: symbolic dimensions refined, ranges narrowed; no shape conflict reaches runtime
  \item \textbf{Semantic subsumption}: Measurements match in frame and unit; Distributions normalize correctly; Enums share vocabulary or use explicit mappings; Structured types align field-by-field
\end{enumerate}
\textbf{Example of semantic fusion:} A vision encoder outputs \texttt{image\_emb : f32[B,512] ⊥}. A language model outputs \texttt{text\_emb : f32[B,512] E\{BPE50k, 50000\}}. Direct concatenation is rejected by the type system because Atomic (\texttt{⊥}) and Enum (\texttt{E}) semantics are incompatible. An explicit bridge morph must be used: \texttt{Embed(text) -> f32[B,512] ⊥}, which projects symbolic knowledge (tokens) into latent space. Only after this projection can the embeddings be fused.

Implicit unit conversion, enum widening, or dimension mismatch is rejected. This preserves auditability: every knowledge realm crossing is explicit and auditable.

\OnePageSectionEnd

\OnePageSectionStart
\section{Sensors as Data Sources}

Sensors are typed producers. The type specifies what semantic information the data carries, enabling deterministic mapping to computation.

\begin{itemize}
  \item \textbf{CameraRGB}: \texttt{f32[B,C,H,W] M\{radiance,srgb,camera,[0,1]\}} \quad Converts pixel radiance to normalized $[0,1]$ RGB.
  \item \textbf{Microphone}: \texttt{f32[B,F,T] M\{amplitude,log\_mel,acoustic,[-1,1]\}} \quad Mel-spectrogram: log-energy features over frequency/time.
  \item \textbf{Tokenizer}: \texttt{i32[B,T] E\{BPE\_50k, 50000\}} \quad Text to token indices.
  \item \textbf{ClassificationLabel}: \texttt{i32[B] E\{ImageNet1k, 1000\}} \quad Train-time: ground-truth class index.
\end{itemize}

Sensor bindings are explicit: \texttt{bind CameraRGB.output $\to$ ConvStem.input} enforces that the downstream morph expects a Measurement with matching unit and frame. Types guarantee \emph{semantic coherence}---every operation respects the data's origin and meaning---but not physical truth. A malfunctioning sensor will produce a tensor of the correct type with incorrect content; the type system ensures correct composition regardless.

\OnePageSectionEnd

\OnePageSectionStart
\section{Typed Morphs}

Morphs are type-preserving transformations. Each morph declares its input and output types; the type checker ensures compatibility at every edge.

\paragraph{Linear Transformation}
\begin{verbatim}
morph Linear(x: f32[B,T,D] ⊥) -> f32[B,T,D_out] {
  wt: W: f32[D, D_out] ⊥
}
\end{verbatim}

\paragraph{Multi-Head Attention}
\begin{verbatim}
morph Attention(q,k,v: f32[B,H,T,D] ⊥) -> f32[B,H,T,D] {
  chk: shape(q,k), shape(k,v)
}
\end{verbatim}

\paragraph{Distribution-Aware Softmax}
\begin{verbatim}
morph Softmax(logits: f32[B,C] ⊥) -> f32[B,C] D{categorical, [0,1]} {
  chk: norm(logits) // enforce probabilistic support
}
\end{verbatim}

Key insight: semantics are explicit. Softmax output is \emph{not} merely normalized values---it is \emph{typed} as a Distribution, enabling downstream morphs (Sample, KL divergence, cross-entropy loss) to enforce semantic correctness. This makes stochastic and supervised pipelines verifiable.

\OnePageSectionEnd

\OnePageSectionStart
\section{Fusing Knowledge Realms: From Observation to Action}

A complete pipeline from observation to action is a DAG where every edge carries a tensor type. The graph fuses three distinct realms of intelligence: \emph{perception} (grounded measurement), \emph{knowledge} (symbolic and probabilistic), and \emph{reasoning} (deterministic and stochastic computation).

\paragraph{Perception Realm: Grounding in Physical Reality}
\begin{itemize}
  \item \textbf{CameraRGB} $\to$ \texttt{f32[B,3,H,W] M\{radiance,srgb,camera,[0,1]\}}—raw pixels are physical measurements
  \item \textbf{Microphone} $\to$ \texttt{f32[B,F,T] M\{amplitude,log\_mel,acoustic,[-1,1]\}}—audio is acoustic measurement
  \item \textbf{Tactile Sensor} $\to$ \texttt{f32[B,N] M\{pressure,pascals,tactile,[0,100k]\}}—touch is force measurement
\end{itemize}
All perception inputs are Measurement (\texttt{M}) types: they ground the computation in physical reality with explicit units, frames, and valid ranges. A malfunctioning sensor produces a Measurement of the correct type with wrong content—the type system ensures this is processed consistently and fails explicitly if constraints are violated.

\paragraph{Knowledge Realm: Symbolic and Probabilistic Abstraction}
\begin{itemize}
  \item \textbf{Tokenizer} $\to$ \texttt{i32[B,T] E\{BPE50k,50000\}}—raw text as symbolic vocabulary
  \item \textbf{Ontology/KG} $\to$ \texttt{i32[B,E] E\{DBpedia,1e9\}}—structured knowledge as symbols
\end{itemize}
Symbolic knowledge is typed as Enum (\texttt{E}): discrete, finite vocabulary. Probabilistic knowledge (language model output, belief states) is typed as Distribution (\texttt{D}): normalized over a support space.

\paragraph{Reasoning Realm: Fusing Knowledge Through Type-Preserving Morphs}
\textbf{Embedding projection}: Bridge symbolic knowledge into latent space:
\begin{itemize}
  \item \texttt{TextEmbed}: \texttt{E\{vocab\}} $\to$ \texttt{⊥} (project tokens to embeddings)
  \item \texttt{EntityEmbed}: \texttt{E\{entities\}} $\to$ \texttt{⊥} (project knowledge graph entities to embeddings)
  \item \texttt{MeasurementNorm}: \texttt{M} $\to$ \texttt{⊥} (normalize physical quantities to standard ranges)
\end{itemize}

\textbf{Fusion across realms}:
\[ \text{X} = \text{Concat}([\text{text\_emb}, \text{vision\_emb}, \text{entity\_emb}, \text{tactile\_emb}], \text{axis}=1) \]
All inputs to Concat are now Atomic (\texttt{⊥}, latent, dimensionless). The type system ensures shape alignment and semantic compatibility. Mixing a Measurement directly with a symbolic Enum is impossible—all knowledge realms must first be projected to a common latent space.

\textbf{Integrated reasoning}:
Apply \emph{N} transformer blocks to fused representation. Each block composes morphs that preserve semantic structure:
\[ \text{X'} = \text{Attention}(\text{X}) \to \text{f32[B,T,D] ⊥} \]
\[ \text{X''} = \text{Linear}(\text{LayerNorm}(\text{X'})) \to \text{f32[B,T,D] ⊥} \]
\[ \text{logits} = \text{Linear}(\text{X''}) \to \text{f32[B,1,C] ⊥} \]
The output logits are still Atomic (\texttt{⊥})—pure latent vectors. No semantic information flows from this layer; it is deterministic (no Distribution) because reasoning over fused knowledge is deterministic until explicitly stochastic.

\textbf{Decision and action}:
Bridge reasoning output back to action in physical and symbolic realms:
\[ \text{action\_logits} = \text{Linear}(\text{X''}) \to \text{f32[B,1,A] ⊥} \]
\[ \text{action\_probs} = \text{Softmax}(\text{action\_logits}) \to \text{f32[B,1,A] D\{categorical, [0,1]\}} \]
Actuators bind to Distribution (\texttt{D}) or Measurement (\texttt{M}) outputs:
\begin{itemize}
  \item \textbf{Motor}: Sample from Distribution, e.g., $\text{action} \sim \text{action\_probs}[\text{``left''},\text{``straight''},\text{``right''}]$
  \item \textbf{Controller}: Project back to physical Measurement, e.g., \texttt{speed\_ctrl: f32[B,1] M\{speed,m/s,motor,[0,10]\}}
\end{itemize}

\textbf{Semantic invariant:} Every edge in the graph is typed. Perception (\texttt{M}) grounds input. Knowledge (\texttt{E}, \texttt{D}) enters via symbolic/probabilistic projections. Reasoning fuses and transforms Atomic (\texttt{⊥}) representations. Decision outputs (\texttt{D}) drive stochastic action; Motor control (\texttt{M}) enforces physical safety. The type system rejects (at compile time) any crossing of knowledge realms without explicit type bridge morphs.

\OnePageSectionEnd

\OnePageSectionStart
\section{Semantic-Preserving Lowering to Runtime}

The typed AST is lowered to any target runtime (ONNX IR, custom GPU kernels, CPU code, hardware accelerators, or domain-specific interpreters) while preserving semantic invariants. Assertions are explicit; runtime checks are elided only when provably safe; modality semantics are preserved in metadata for downstream auditing and validation.

\paragraph{Type $\to$ Runtime Mapping}
\begin{center}
\begin{tabular}{lll}
\hline
\textbf{Typed Tensor} & \textbf{Runtime Encoding} & \textbf{Assertions} \\
\hline
Measurement (\texttt{M}) & Tensor + metadata & Clip, Assert on range and unit \\  
Distribution (\texttt{D}) & float32/float64 tensor & \emph{normalization}; constraint is semantic \\
Enum (\texttt{E}) & int32 tensor & Assert bounds and vocabulary \\  
Atomic (\texttt{⊥}) & float32/float64 tensor & generic numeric range \\
\hline
\end{tabular}
\end{center}

\textbf{ONNX as default target}: ONNX IR provides an extensible format for intermediate representation. Tensors are lowered to ONNX tensors with semantic metadata stored as attributes or external schema references. Assertions (on Measurement ranges, Enum bounds, Distribution normalization) are encoded as ONNX op graphs (Clip, Greater, Assert nodes).

\textbf{Alternative runtimes}: The same typed AST can lower to GPU-optimized code (CUDA kernels with semantic metadata in comments), CPU code (with explicit runtime checks inlined), or specialized accelerators (custom instruction sequences with semantic annotations).

\paragraph{Assertion Elision and Symbolic Analysis}
Compile-time shape and range analysis determines which runtime assertions are redundant:
\begin{itemize}
  \item If range $[0,1]$ is guaranteed by prior Softmax, subsequent Clip is elided.
  \item If Enum bounds are verified statically, vocabulary check is removed.
  \item If symbolic dimension is never refined, shape Assert remains at runtime.
\end{itemize}

\paragraph{Semantic Preservation and Auditability}
Regardless of target runtime, the lowered code carries semantic annotations for every tensor:
\begin{itemize}
  \item \textbf{Auditing}: trace back any output to its sensor input and all intermediate semantic types.
  \item \textbf{Test/Train Consistency}: graph topology and type constraints are deterministic and runtime-agnostic; train and test graphs share identical lowered semantics.
  \item \textbf{Model Debugging}: when output is unexpected, the semantic types and assertions narrow culprit to a specific morph, knowledge realm crossing, or measurement constraint violation.
  \item \textbf{Runtime Flexibility}: same ABI/AST can be deployed on different hardware (edge device with lightweight interpreter, cloud GPU with ONNX Runtime, custom accelerator with custom lowering) without changing the semantic contract.
\end{itemize}

\OnePageSectionEnd

\OnePageSectionStart
\section{EBNF Grammar of the AST}

The type-safe ABI is defined by a formal grammar. Tensors, morphs, and graphs are syntactic objects subject to type checking before lowering.

\subsection{Type Grammar}
\begin{verbatim}
Type ::= TensorType | MorphRef | GraphRef

TensorType ::= DType Shape [Sem]

DType ::= f16 | f32 | f64 | i8 | i32 | i64 | bool

Shape ::= "[" Dim { "," Dim } "]"
Dim   ::= Number | Ident | "?" | Range
Range ::= Number ".." Number

Sem ::= "⊥"
      | "M" "{" Qty "," Unit "," Frame "," "[" Number "," Number "]" "}"
      | "D" "{" Kind "," Support "}"
      | "E" "{" Vocab "," Number "}"
      | "{" Field { "," Field } "}"

Field ::= Ident ":" Type
\end{verbatim}

\subsection{Morph Grammar}
Morphs are typed transformations (morphisms) with declared signatures and internal constraints:
\begin{verbatim}
MorphDef ::= morph Ident Sig MorphBlock

Sig ::= "(" [ Param { "," Param } ] ")" "->" Type
Param ::= Ident ":" Type

MorphBlock ::= "{"
                 { ConstDecl }
                 { WeightDecl }
                 { CheckDecl }
               "}"

ConstDecl  ::= "const" ":" Binding { "," Binding }
WeightDecl ::= "wt"    ":" Binding { "," Binding }
Binding    ::= Ident ":" Type

CheckDecl  ::= "chk" ":" Constraint { "," Constraint }

Constraint ::= shape(Ident, Ident)
            | dtype(Ident, Ident)
            | sem_sub(Ident, Ident)
            | range(Ident, [Number, Number])
            | norm(Ident)
\end{verbatim}

\subsection{Graph Grammar}
\begin{verbatim}
GraphDef ::= graph Ident "(" ")" GraphBlock

GraphBlock ::= "{"
                 { PortDecl }
                 { NodeDecl }
                 { EdgeDecl }
               "}"

PortDecl ::= in ":" Binding { "," Binding }
           | out ":" Binding { "," Binding }

NodeDecl ::= N ":" Node { "," Node }
Node     ::= Ident ":" (MorphRef | "Const")

EdgeDecl ::= E ":" Edge { "," Edge }
Edge ::= "(" Ref ")" "->" "(" Ref ")" [ ":" Type ]
Ref  ::= Ident "." Ident
\end{verbatim}

\OnePageSectionEnd

\OnePageSectionStart
\section{Semantic Foundations and RDF Grounding}

Semantics are not merely informal annotations---they are first-class objects in the type system. We ground semantics in RDF (Resource Description Framework) and IRIs (Internationalized Resource Identifiers), establishing a global namespace for meaning.

\subsection{Semantic Types as RDF Entities}
Each semantic class (Measurement, Distribution, Enum, Structured) is instantiated as an RDF class, enabling interoperability and formal reasoning:

\begin{verbatim}
Measurement rdf:type owl:Class
  rdfs:subClassOf Semantic
  dcterms:hasPhysicalQuantity IRI
  qudt:hasUnit IRI
  sosa:hasResultTime xsd:dateTime
  sosa:isObservedBy IRI  // sensor/observer IRI
  sosa:madeBySensor IRI
\end{verbatim}

Example instantiation:
\begin{verbatim}
ex:camera_radiance rdf:type Measurement
  ex:quantityOf ex:Radiance
  qudt:unit qudt:Watt-Per-Steradian-Per-Meter_Squared
  sosa:madeBySensor ex:FrontCamera
  ex:frame ex:CameraFrame_320x240
\end{verbatim}

\subsection{Enum Semantics: Symbolic Knowledge as Controlled Vocabularies}
Enum types map to SKOS (Simple Knowledge Organization System) concept schemes:

\begin{verbatim}
ex:BPE50kVocab rdf:type skos:ConceptScheme
  skos:prefLabel "BPE 50k Vocabulary"
  skos:hasTopConcept [multiple BPE token IRIs]

ex:ImageNet1kClasses rdf:type skos:ConceptScheme
  skos:prefLabel "ImageNet 1000 Classes"
  skos:hasTopConcept [1000 class IRIs]
\end{verbatim}

Each token/class is an IRI pointing to semantic meaning in a knowledge graph (WordNet, DBpedia, or domain-specific ontologies).

\subsection{Distribution Semantics: Probabilistic Grounding}
Distribution types carry semantic information about what is being modeled:

\begin{verbatim}
Distribution rdf:type owl:Class
  ex:overId IRI          // IRI of entity being modeled
  ex:supports IRI        // support space IRI
  ex:isNormalized xsd:boolean
  ex:hasStochasticProcess IRI  // Poisson, Gaussian, etc.
\end{verbatim}

Example:
\begin{verbatim}
ex:action_distribution rdf:type Distribution
  ex:overId ex:DiscreteActions
  ex:supports ex:ActionSpace_3ways
  ex:isNormalized true
  // Softmax output over {left, straight, right}
\end{verbatim}

\subsection{The Fabric of Meaning: Semantic Graph Structure}
Semantics form a graph structure (the ``fabric of meaning'') orthogonal to the computation graph:

\begin{itemize}
  \item **Nodes**: RDF entities (measurements, symbols, distributions, physical quantities)
  \item **Edges**: RDF predicates (``unit of'', ``instance of'', ``hasStochasticProcess'', ``observedBy'')
  \item **Namespace**: Global IRIs enable linking across systems and domains
\end{itemize}

When a morph transforms a tensor, it must preserve semantic coherence: if input is a Measurement with ``unit: Kelvin'', output must either remain Kelvin or use an explicit conversion semantic (e.g., to Celsius via \texttt{owl:hasConversion}).

This fabric enables:
\begin{itemize}
  \item \textbf{Semantic interoperability}: two systems using same IRIs can verify semantic compatibility
  \item \textbf{Ontology-driven verification}: reason over semantic constraints (e.g., cannot mix Celsius and Kelvin without conversion)
  \item \textbf{Formal grounding}: semantics are not strings but globally resolvable IRIs with formal definitions
\end{itemize}

\OnePageSectionEnd

\OnePageSectionStart
\section{Formal Semantics and Graph Theory}

\subsection{Denotational Semantics of Typed Tensors}
A typed tensor is interpreted as a pair: value and semantic grounding.

\[ \llbracket \text{Tensor}\langle D, S, \Sigma \rangle \rrbracket = (v, \sigma) \]

where $v \in \mathbb{R}^{D}$ (the value) and $\sigma \in \text{SemanticSpace}$ (the meaning).

\textbf{Semantic spaces}:
\begin{itemize}
  \item $\text{Atomic}: \emptyset$ (no semantic constraint)
  \item $\text{Measurement}: (\text{PhysicalQuantity}, \text{Unit}, \text{Frame}) \times [a, b]$ (grounded in physical reality with bounds)
  \item $\text{Distribution}: (\text{Entity}, \text{StochasticProcess}, \text{Support})$ (probabilistic model of an entity)
  \item $\text{Enum}: \text{ConceptScheme} \times \mathbb{N}$ (symbolic vocabulary with cardinality)
  \item $\text{Structured}: \prod_i \text{SemanticSpace}_i$ (product of semantic spaces)
\end{itemize}

A morph $M$ is a morphism that preserves semantic structure:
\[ P: (v_1, \sigma_1) \times \cdots \times (v_n, \sigma_n) \to (v'_1, \sigma'_1) \times \cdots \times (v'_m, \sigma'_m) \]

\textbf{Semantic preservation constraint}: $P$ must satisfy $\sigma'_j \sqsubseteq \sigma_j$ (semantic subsumption) or involve an explicit type bridge.

\subsection{Graph-Theoretic Model of Computation}
A computation graph $G = (N, E, \tau)$ where:
\begin{itemize}
  \item $N$ is a set of morph nodes
  \item $E \subseteq N \times N$ are directed edges (data flow)
  \item $\tau: E \to \text{Type}$ assigns a type to each edge
\end{itemize}

The graph must satisfy:
\begin{enumerate}
  \item \textbf{Well-typedness}: for each edge $(u, v) \in E$, output type of $u$ is compatible with input type of $v$
  \item \textbf{Acyclicity}: no cycles (data flow is a DAG)
  \item \textbf{Semantic closure}: semantic types form a lattice under subsumption; each edge type is a lattice element
\end{enumerate}

\subsection{Three Orthogonal Graph Layers}

A statically typed system actually operates over three intertwined graphs:

\paragraph{1. Computation Graph}
The data flow DAG: $G_{\text{comp}} = (N_{\text{prim}}, E_{\text{data}})$ where edges carry tensor types. This is what executes at runtime.

\paragraph{2. Type Graph}
The constraint system over types: $G_{\text{type}} = (N_{\text{type}}, E_{\text{subtype}})$ where nodes are types and edges are subsumption relations. The system checks that every data edge respects type constraints.

\paragraph{3. Semantic Graph}
The RDF/ontology layer: $G_{\text{sem}} = (N_{\text{iri}}, E_{\text{predicate}})$ where nodes are IRIs (physical quantities, knowledge concepts, stochastic processes) and edges are RDF predicates. This ground-truth graph describes what meanings the computation is over.

\textbf{Invariant}: For every edge $e \in E_{\text{data}}$ with type $\tau(e)$, the semantics of $\tau(e)$ must resolve via $G_{\text{sem}}$. If $\tau(e) = \text{Measurement}\{\text{velocity}, \text{unit:m/s}\}$, then $\text{velocity}$ and $\text{m/s}$ must be IRIs resolvable in $G_{\text{sem}}$.

\subsection{The Fabric of Reality, Knowledge, and Computation}

These three graphs form a unified fabric:
\begin{itemize}
  \item **Reality Fabric** ($G_{\text{sem}}$ layer): the ontology of physical quantities, measured by sensors
  \item **Knowledge Fabric** ($G_{\text{sem}}$ layer): the vocabulary of symbols and concepts (tokens, entities, relations)
  \item **Computation Fabric** ($G_{\text{comp}}$ layer): the neural network DAG that transforms observations
  \item **Type Fabric** ($G_{\text{type}}$ layer): the constraints ensuring safe composition
\end{itemize}

These fabrics are not separate---they are woven together. A Measurement tensor carries both:
- A value (computation fabric): the pixel values from a camera
- A semantic IRI (reality fabric): the physics quantity ``radiance'' with unit ``sRGB''

A Distribution tensor carries:
- A value (computation fabric): probabilities summing to 1
- A semantic IRI (knowledge fabric): the decision space it models

When morphs compose, all three fabrics must align. This is why semantic typing is powerful: it enforces alignment across all three simultaneously.

\OnePageSectionEnd

\OnePageSectionStart
\section{Related Work}

\subsection{Type Systems and Dependent Types}
Type systems in programming languages (Hindley-Milner \citep{hindley1969}, System F \citep{girard1972}) provide shape and kind discipline. Recent work on dependent types (Coq \citep{coq2023}, Agda \citep{agda2023}) enables richer specifications where types encode propositions. Our semantic types extend this paradigm: instead of properties of computation, they encode properties of \emph{meaning}. A Measurement type is not a computability statement but a grounding statement: ``this tensor represents physical reality.''

Array languages (Futhark \citep{futhark2016}, Dex \citep{dex2022}) provide shape polymorphism and compile-time shape inference, but treat arrays as untyped numeric substrates. They lack semantic distinction for physical quantities, distributions, or symbolic data. Julia's type system \citep{bezanson2017} is flexible but permissive; no constraint on semantic meaning flows through composition.

\subsection{Tensor Type Systems and Neural Language Design}
TensorFlow's shape inference \citep{tensorflow2016} and PyTorch's type hints \citep{pytorch2019} provide partial static guarantees but remain permissive: dynamic shapes, implicit semantic assumptions, no end-to-end verification. Glow \citep{glow2020}, a compiler for neural networks, performs shape inference and code generation but lacks semantic types.

More recent work on certified neural networks (VNN Abstr. \citep{neurips2020}, DeepTest \citep{deeptest2018}) focuses on robustness properties (adversarial bounds) rather than semantic grounding. Our semantic types are orthogonal to robustness but enable semantic verification at a higher level than floating-point bounds.

\subsection{RDF, Ontologies, and Semantic Web}
RDF \citep{rdf2014}, OWL \citep{owl2012}, and SKOS \citep{skos2009} provide formal frameworks for describing meaning via IRIs. QUDT \citep{qudt2023} (Quantities, Units, Dimensions, Types) defines ontologies for physical units and dimensions. SOSA/SSN \citep{sosa2017} (Sensor, Observation, Sample, and Actuator) is a W3C standard for sensor data and observations.

Our contribution is to embed this semantic infrastructure \emph{into the tensor type system}. Rather than treating semantics as external metadata, we make them first-class type properties that drive compilation and verification. This bridges formal ontology languages with neural computation.

\subsection{Domain-Specific Languages for Neural Computation}
Halide \citep{halide2013} provides scheduling and optimization for image pipelines, but lacks neural primitives. Darkroom \citep{darkroom2013} uses dependent types for image processing but is not neural-focused. TVM \citep{tvm2018} is a compiler for diverse backends (CPU, GPU, TPU) but operates on untyped tensor IRs.

MLIR \citep{mlir2021} provides a framework for intermediate representations and lowering. Our work can be viewed as a domain-specific dialect of MLIR, where the semantic type system is a core abstraction.

\subsection{Knowledge Graphs and Neurosymbolic Reasoning}
Knowledge graphs (Freebase \citep{freebase2008}, DBpedia \citep{dbpedia2014}, Wikidata \citep{wikidata2014}) enable structured reasoning over symbolic knowledge. Neurosymbolic AI (Mao et al. \citep{neurosymbolic2019}, Garnelo and Shanahan \citep{neurosymbolic2019b}) attempts to integrate neural and symbolic reasoning.

Our semantic fabric (Section 9) connects neural tensors directly to KG IRIs, enabling compositional verification that fuses neural and symbolic reasoning without ad-hoc glue code. When an Enum tensor references a SKOS vocabulary, the type system can verify that operations respect vocabulary constraints.

\subsection{Formal Semantics of Machine Learning Systems}
Recent work on formal specification of ML systems (Uchitel et al. \citep{ml_formalism2021}, Heil et al. \citep{ml_semantics2022}) proposes formal models of training dynamics and generalization. Our approach is complementary: we focus on \emph{graph-level} semantics (what computation means) rather than \emph{statistical} semantics (what generalization guarantees hold).

\subsection{Verified Compilers and Certified Code Generation}
CompCert \citep{compcert2023}, a fully verified C compiler, proves correctness of code generation. CakeML \citep{cakeml2023} extends this to functional languages. Our work borrows the philosophy---semantics preservation through lowering is verified or at least auditable---but applies it to neural computation and semantic (not just syntactic) correctness.

\subsection{Robotics and Autonomous Systems}
ROS (Robot Operating System) \citep{ros2013} provides typed message definitions and sensor abstractions. Gazebo \citep{gazebo2020} simulates physical systems. Our work complements these by providing compile-time verification that perception, reasoning, and action are semantically aligned. A typical ROS node might operate on untyped sensor messages; our ABI/AST would provide static guarantees that sensor frames, units, and action ranges are respected throughout.

\subsection{Our Distinguishing Contributions}
\begin{enumerate}
  \item \textbf{Semantic Types as First-Class}: unlike prior work, semantics (Measurement, Distribution, Enum) are not annotations but type-system objects that drive verification and lowering.
  \item \textbf{RDF/IRI Grounding}: we explicitly tie tensor semantics to global RDF IRIs, enabling interoperability and ontology-driven reasoning.
  \item \textbf{Three-Graph Model}: we formalize the interplay of computation, type, and semantic graphs, showing how all three must align for a well-formed system.
  \item \textbf{Knowledge Realm Fusion}: we provide explicit type-safe bridges for crossing between perception (Measurement), knowledge (Enum/Distribution), and reasoning (Atomic), preventing silent semantic drift.
  \item \textbf{Runtime Agnostic}: unlike ONNX-specific approaches, our ABI/AST lowers to any target while preserving semantics.
\end{enumerate}

\OnePageSectionEnd

\OnePageSectionStart
\section{Type Checking and Guarantees}

\subsection{Compile-Time Checks}
The type checker validates:
\begin{enumerate}
  \item \textbf{Shape Unification}: symbols refined, ranges narrowed; no shape conflict reaches runtime.
  \item \textbf{DType Compatibility}: no implicit numeric casts; morphs declare exact DType.
  \item \textbf{Semantic Coherence}: Measurement units match; Distribution-aware ops are applied only to Distribution types; Enum vocabularies align.
\end{enumerate}
Rejects ill-typed programs statically. Where checks cannot be discharged (e.g., dynamic sequence length), emits explicit runtime Assert.

\subsection{Formal Properties}
\begin{itemize}
  \item \textbf{Progress}: \emph{Well-typed graphs execute without shape or DType errors.} (Proof: by induction over type check, every morph's output matches successor's input.)
  \item \textbf{Semantic Preservation}: \emph{Tensor semantics flow through morphs.} Measurement frame is preserved; Distribution normalization is enforced; Enum bounds are maintained. No implicit unit conversion or categorical collapse.
  \item \textbf{Scope}: Types ensure internal compositional correctness, not physical ground truth. A sensor may malfunction, but the type system ensures the erroneous data is processed consistently and fails explicitly if constraints are violated.
\end{itemize}

\OnePageSectionEnd

\OnePageSectionStart
\section{Code Synthesis and Automation}

A declarative ABI enables deterministic code generation by LLMs and automated tools. Sensor bindings, pipeline wiring, and safety wrappers (assertions, bounds checks) can be synthesized from high-level specifications. Example:

\paragraph{From spec to pipeline}
An LLM given a type schema (sensor outputs, model architecture, actuator constraints) can generate:
\begin{itemize}
  \item Sensor binding code: \texttt{bind CameraRGB.output -> ConvStem.input} with type validation
  \item Graph interconnect: ensuring shape/semantic compatibility at each edge
  \item Safety wrappers: explicit assertions for dynamic ranges, with justification for which checks are elided
\end{itemize}

This reduces manual engineering effort and improves reproducibility: two teams using the same type schema will generate semantically equivalent pipelines.

\OnePageSectionEnd

\OnePageSectionStart
\section{Discussion}

\subsection{Scope and Ground Truth}
Types guarantee \emph{logical} consistency: every morph respects its input types, and output types are verifiably correct. They do \emph{not} guarantee \emph{physical} truth. A camera sensor may be misaligned, low-light, or occluded; the resulting tensor will still have type \texttt{f32[B,C,H,W] M\{radiance,srgb,camera\}} even if content is corrupted. Empirical validation (train/test splits, unit tests, sensor calibration) remains essential. The type system makes failures \emph{explicit and auditable}: when a sensor or morph produces surprising output, the type system preserves evidence (types, semantics, bounds) enabling root-cause diagnosis.

\subsection{Extensibility}
The ontology of Semantics (Atomic, Measurement, Distribution, Enum, Structured) is intentionally minimal but extensible. Rich unit systems (SI, custom), physical constraints, and domain-specific types (e.g., poses, graph structures) can be added without breaking the core type checker.

\subsection{Performance}
Assertion elision via symbolic analysis and range inference minimizes runtime overhead. For large models, the cost of metadata and type checks is amortized across thousands of operations. Benchmarking against untyped baselines (raw ONNX) is needed; preliminary evidence suggests overhead $<5\%$ with aggressive elision.

\subsection{Test/Train Reproducibility}
The type system makes test and train graphs identical in structure and semantics. Determinism is enforced: same sensor input + same model weights $\Rightarrow$ same output (modulo floating-point rounding). This enables reproducible evaluation, ablation studies, and formal validation—critical for safety-critical AI.

\OnePageSectionEnd

\OnePageSectionStart
\section{Conclusion}

We present a statically typed ABI and AST that grounds machine intelligence in reality by fusing heterogeneous knowledge realms. Sensors produce Measurement types (physical reality). Symbolic knowledge enters as Enum types. Probabilistic models output Distribution types. Type-preserving morphs enable safe composition and explicit projection between realms. The ABI lowers to any runtime---ONNX, custom kernels, specialized hardware---ensuring semantic invariants are preserved end-to-end.

Test and train graphs are identical in structure and semantics, enabling reproducible validation. Compile-time type checking rejects semantic unsafe programs; runtime assertions are explicit and elided when provably safe. By embedding DType, Shape, and Semantics into tensors, we achieve deterministic mapping of observations to computation to action, enabling machines to interpret the world with formal, verifiable reasoning grounded in measurable reality.

Future work includes richer semantic ontologies (physical unit systems, stochastic process types), certified compiler backends with formal proofs of semantic preservation, and integration with symbolic reasoning systems to bridge discrete and continuous knowledge.

\OnePageSectionEnd


% Bibliography
\clearpage
\bibliography{bib/references}

\end{document}
