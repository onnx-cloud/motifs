\OnePageSectionStart
\section{Semantic Foundations and RDF Grounding}

Semantics are not merely informal annotations---they are first-class objects in the type system. We ground semantics in RDF (Resource Description Framework) and IRIs (Internationalized Resource Identifiers), establishing a global namespace for meaning.

\subsection{Semantic Types as RDF Entities}
Each semantic class (Measurement, Distribution, Enum, Structured) is instantiated as an RDF class, enabling interoperability and formal reasoning:

\begin{verbatim}
Measurement rdf:type owl:Class
  rdfs:subClassOf Semantic
  dcterms:hasPhysicalQuantity IRI
  qudt:hasUnit IRI
  sosa:hasResultTime xsd:dateTime
  sosa:isObservedBy IRI  // sensor/observer IRI
  sosa:madeBySensor IRI
\end{verbatim}

Example instantiation:
\begin{verbatim}
ex:camera_radiance rdf:type Measurement
  ex:quantityOf ex:Radiance
  qudt:unit qudt:Watt-Per-Steradian-Per-Meter_Squared
  sosa:madeBySensor ex:FrontCamera
  ex:frame ex:CameraFrame_320x240
\end{verbatim}

\subsection{Enum Semantics: Symbolic Knowledge as Controlled Vocabularies}
Enum types map to SKOS (Simple Knowledge Organization System) concept schemes:

\begin{verbatim}
ex:BPE50kVocab rdf:type skos:ConceptScheme
  skos:prefLabel "BPE 50k Vocabulary"
  skos:hasTopConcept [multiple BPE token IRIs]

ex:ImageNet1kClasses rdf:type skos:ConceptScheme
  skos:prefLabel "ImageNet 1000 Classes"
  skos:hasTopConcept [1000 class IRIs]
\end{verbatim}

Each token/class is an IRI pointing to semantic meaning in a knowledge graph (WordNet, DBpedia, or domain-specific ontologies).

\subsection{Distribution Semantics: Probabilistic Grounding}
Distribution types carry semantic information about what is being modeled:

\begin{verbatim}
Distribution rdf:type owl:Class
  ex:overId IRI          // IRI of entity being modeled
  ex:supports IRI        // support space IRI
  ex:isNormalized xsd:boolean
  ex:hasStochasticProcess IRI  // Poisson, Gaussian, etc.
\end{verbatim}

Example:
\begin{verbatim}
ex:action_distribution rdf:type Distribution
  ex:overId ex:DiscreteActions
  ex:supports ex:ActionSpace_3ways
  ex:isNormalized true
  // Softmax output over {left, straight, right}
\end{verbatim}

\subsection{The Fabric of Meaning: Semantic Graph Structure}
Semantics form a graph structure (the ``fabric of meaning'') orthogonal to the computation graph:

\begin{itemize}
  \item **Nodes**: RDF entities (measurements, symbols, distributions, physical quantities)
  \item **Edges**: RDF predicates (``unit of'', ``instance of'', ``hasStochasticProcess'', ``observedBy'')
  \item **Namespace**: Global IRIs enable linking across systems and domains
\end{itemize}

When a morph transforms a tensor, it must preserve semantic coherence: if input is a Measurement with ``unit: Kelvin'', output must either remain Kelvin or use an explicit conversion semantic (e.g., to Celsius via \texttt{owl:hasConversion}).

This fabric enables:
\begin{itemize}
  \item \textbf{Semantic interoperability}: two systems using same IRIs can verify semantic compatibility
  \item \textbf{Ontology-driven verification}: reason over semantic constraints (e.g., cannot mix Celsius and Kelvin without conversion)
  \item \textbf{Formal grounding}: semantics are not strings but globally resolvable IRIs with formal definitions
\end{itemize}

\OnePageSectionEnd
