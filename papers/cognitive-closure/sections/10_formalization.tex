\OnePageSectionStart
\section{Formal Proofs and Formalization Roadmap}

This section outlines the formalization artifacts accompanying Cognitive Closure and presents proof sketches for the principal claims.

\subsection{Projection Properties}
We include a Coq proof skeleton \texttt{proofs/typed_projection.v} that establishes the following lemmas for the box projection operator: (1) boundedness (post-projection lies within the defined interval), and (2) non-expansiveness (projection is non-increasing in distance to any feasible point). These lemmas are critical building blocks for proving convergence of Projected Gradient methods in the Fused Fabric.

\begin{theorem}[Typed PGD Convergence Sketch]
Let $\Sigma$ be a closed, convex semantic constraint set and let $\mathcal{L}$ be a $L$-smooth function on the parameter space. Consider the typed projected gradient descent updates:
\[ \theta_{t+1} = \Pi_{\Sigma}(\theta_t - \eta \nabla \mathcal{L}(\theta_t)). \]
If $0 < \eta < 1/L$, then the sequence $\{\theta_t\}$ satisfies
\[ \min_{0\le t < T} \|\nabla_{\Sigma} \mathcal{L}(\theta_t)\|^2 = O\left(\frac{\mathcal{L}(\theta_0)-\mathcal{L}^*}{\eta T}\right), \]
where $\nabla_{\Sigma}$ denotes the projected gradient and $\mathcal{L}^*$ is the infimum of $\mathcal{L}$ on $\Sigma$.
\end{theorem}
\begin{proof}[Sketch]
The proof follows standard PGD analyses (Nocedal & Wright \citep{optimization2006}). Key observations: projections onto closed convex sets are non-expansive, and the descent lemma for $L$-smooth functions yields a sufficient decrease per step when the step-size satisfies $\eta < 1/L$. Combining these with the non-expansiveness of $\Pi_{\Sigma}$ yields the asymptotic bound above. Crucially, our Coq formalization targets the projection lemmas and the descent lemma to mechanize this sketch.
\end{proof}

\subsection{Adjoint Correctness}
We state a correctness criterion for the adjoint graph generated by the compiler.
\begin{definition}[Adjoint Equivalence]
Let $G$ be the primal fused graph and $G^*$ be the adjoint graph produced by the compiler via AD. $G^*$ is adjoint-equivalent to $G$ if, for any parameter perturbation $\Delta \theta$, the finite-difference approximation of the primal perturbation equals the adjoint evaluation:
\[ \lim_{\epsilon \to 0} \frac{\mathcal{L}(\theta+\epsilon\Delta\theta)-\mathcal{L}(\theta)}{\epsilon} = \langle \nabla_\theta \mathcal{L}(\theta), \Delta\theta \rangle = G^*(\Delta\theta). \]
\end{definition}

\paragraph{Verification strategy}
We verify adjoint correctness by combining symbolic differentiation of small motifs (e.g., LOF) with execution equivalence tests on the adjoint graph. Mechanized proofs will proceed by induction on the operator set and use equational reasoning to show that the AD rules preserve the inner-product characterization of directional derivatives.

\subsection{Complexity and Overhead}
Generation of adjoint graphs increases IR size (roughly by a factor depending on operator arity and statefulness). Projection operations add a modest computational overhead (typically linear in parameter size for box/simplex projections). We argue that this overhead is justified by improved semantic safety and often reduced iteration counts due to constrained optimization.

\subsection{Formal Artifacts}
Present artifacts include:
\begin{itemize}
  \item \texttt{proofs/typed\_projection.v}: Coq skeleton proving box projection lemmas and non-expansiveness.
  \item \texttt{proofs/adjoint\_sketches.md}: notes and invariants for adjoint correctness proofs.
  \item RDF snippets encoding common semantic constraints for verifier consumption.
\end{itemize}

\OnePageSectionEnd