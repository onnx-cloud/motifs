\OnePageSectionStart
\section{Silicon-Native Execution}

By targeting the machine-native ONNX format, the Cognitive Compiler bridges the gap between reasoning and hardware.

\paragraph{Universal Backend Support}
The compiled artifacts can be executed on any standard runtime, including ONNX Runtime, OpenVINO, and CoreML. This eliminates the need for specialized "cognitive hardware" and allows reasoning to scale with standard GPU/NPU performance.

\paragraph{Weights as Governed Assets}
Weights and learned parameters are valuable assets: they can be costly to produce, proprietary, or subject to regulatory or contractual use constraints. The Cognitive Compiler treats weights as external artifacts referenced by IRIs with attached metadata (provenance, licensing, cost, and access-control policies). Before composing or deploying a fused fabric, the compiler can (1) query availability and license compatibility, (2) plan alternative computation when weights are unavailable (e.g., untrained transforms, distilled approximations, or hardware-safe replacements), and (3) annotate ModelProto with weight-asset metadata for runtime enforcement and auditing.

\paragraph{Deterministic Low-Level Logic}
Because every operation is statically typed and lowered to exact machine ops, the execution is bit-deterministic. This is critical for industrial applications where "black box" behavior is unacceptable and every decision must be repeatable.

\paragraph{Human Oversight and Machine Reasoning}
The design separates algorithmic structure (the transform graph) from the parameter assets (weights), enabling independent governance: humans can audit and approve computation pipelines, while automated reasoners can verify license constraints, check semantic compatibility, and suggest alternative compositions when necessary. Together, this enables mixed human-machine workflows where policy, provenance, and performance trade-offs are explicitly modeled in the compilation pipeline.

\OnePageSectionEnd
