\OnePageSectionStart
\section{The RDF Meta-Model of Intelligence}

The Fused Fabric is natively described using the Resource Description Framework (RDF) and Web Ontology Language (OWL). This allows the fabric itself to be a queryable knowledge graph.

\paragraph{Self-Describing Intelligence}
By representing the compute graph in RDF, we enable the machine to reason about its own reasoning. Every node in the ONNX graph points back to a URI in an ontology, allowing an auditor to query: "Which physical sensors contributed to this decision, and under what semantic constraints?".

\paragraph{Taxonomy of Morphs}
We use SKOS to categorize reasoning patterns (e.g., \texttt{motif:Linear}, \texttt{motif:Attention}, \texttt{motif:Fusion}). This taxonomy enables the compiler to recognize "reasoning motifs" across different models and replace them with hardware-verified, optimized implementations.

\paragraph{Provenance and Traceability}
The OPM (Open Provenance Model) is used to track the "lineage of reason." When an output tensor is produced, its RDF metadata contains the full graph-fusion history, grounding the inference in a chain of verifiable ontological transitions \citep{rdf2014}.

\paragraph{Weights and Asset Metadata}
Weights and learned parameters are often costly to produce and may be proprietary or subject to licensing restrictions. In our RDF meta-model, weights are first-class assets: they are referenced by IRIs, include provenance, cost/licensing metadata, and may encode access-control constraints. This allows composition queries and auditors to reason about weight availability, license compatibility, and economic cost before a fused fabric is assembled or deployed. When weights are not available or are restricted, the synthesis pipeline may select alternative typed computations (e.g., untrained structure, weightless operators, or lightweight approximations) while preserving semantic typing and formal guarantees.

\subsection{Deterministic Composition and Machine-Assisted Fabric Synthesis}
SPARQL's \texttt{CONSTRUCT} queries provide a deterministic and declarative mechanism to synthesize fused RDF graphs from ontology-aligned building blocks. Because \texttt{CONSTRUCT} is purely declarative, it is an auditable transform: the same input triples and the same query produce the same constructed graph, enabling reproducible composition.

This property enables a practical workflow where programmatic systems (including powerful code-generating language models) can:
\begin{itemize}
  \item Propose Fuse/EBNF fragments that express intended transforms (one-shot generation),
  \item Query the ontology to retrieve compatible motifs and constraints, and
  \item Execute \texttt{CONSTRUCT} templates to deterministically assemble a candidate cognitive fabric that is immediately verifiable by SHACL shapes and type checks.
\end{itemize}

Because every constructed triple carries provenance, the entire synthesis is auditable: a human reviewer can inspect the provenance chain, verify semantic assertions, and accept or amend the constructed fabric prior to lowering.

\OnePageSectionEnd
