\OnePageSectionStart
\section{The Dual Fabrics of Machine Intelligence}

Intelligence requires both a map of the world and a mechanism to navigate it. We formalize these as the Reality Fabric ($\RealityFabric$) and the Cognitive Fabric ($\CognitiveFabric$).

\subsection{The Reality Fabric ($\RealityFabric$)}
The Reality Fabric is defined by the set of all possible grounded observations. Following the principles of SOSA/SSN \citep{sosa2017} and QUDT \citep{qudt2023}, $\RealityFabric$ is a structured space of physical quantities, units, frames of reference, and symbolic concept schemes. A point in $\RealityFabric$ is a \Tensor{DType}{Shape}{Semantics}. 

\subsection{The Cognitive Fabric ($\CognitiveFabric$)}
The Cognitive Fabric is the set of all possible compute graphs that transform information. $\CognitiveFabric$ is defined by the topology of dataflow. Reasoning in $\CognitiveFabric$ is the act of \emph{graph fusion}: taking specialized subgraphs (e.g., a Vision transformer, a Bayesian filter, a Logic gate) and composing them into a globally consistent execution plan.

\subsection{Fusion of Fabrics}
The \emph{Fused Fabric} ($\FusedFabric$) is the product of these two spaces. In this unified realm, every node in a compute graph is not just an operator but a \emph{semantic transform}. An edge in the graph is not just a data tensor but a \emph{grounded fact} flowing between ontological coordinates.

\OnePageSectionEnd
